\usepackage[dvipsnames]{xcolor}
\usepackage{listings}
\usepackage{alloy-style}
\input{commandsFile}
%\documentclass[12pt]{article}
\usepackage[english]{babel}
\usepackage{natbib}
\usepackage{url}
%\usepackage[utf8x]{inputenc}
%\usepackage{amsmath}
%\usepackage{graphicx}
%\graphicspath{{images/}}
%\usepackage{parskip}
%\usepackage{fancyhdr}
%\usepackage{vmargin}
%\setmarginsrb{3 cm}{2.5 cm}{3 cm}{2.5 cm}{1 cm}{1.5 cm}{1 cm}{1.5 cm}

\title{TrackMe Requirement Analysis and Specification Document}								% Title
\author{Andrea Biscontini, Marco Gelli, Alvise De Faveri}								% Author
\date{11 Nov 2018}											% Date

\makeatletter
\let\thetitle\@title
\let\theauthor\@author
\let\thedate\@date
\makeatother

\pagestyle{fancy}
\fancyhf{}
\cfoot{\thepage}

\begin{document}
%TITLE PAGE

\title{2} 

%%%%%%%%%%%%%%%%%%%%%%%%%%%%%%%%%%%%%%%%%%%%%%%%%%%%%%%%%%%%%%%%%%%%%%%%%%%%%%%%%%%%%%%%%

\begin{titlepage}
	\centering
    \vspace*{0.5 cm}
    \includegraphics[scale = 0.75]{Images/Polimilogo}\\[1.0 cm]	% University Logo
    \textsc{\LARGE MSc in Computer Science and Engineering}\\[2.0 cm]	% University Name
	\textsc{\Large Software Engineering 2 Project}\\[0.5 cm]				% Course Code
	\rule{\linewidth}{0.2 mm} \\[0.4 cm]
	{ \huge \bfseries \thetitle}\\
	\rule{\linewidth}{0.2 mm} \\[1.5 cm]
	
	\begin{minipage}{0.4\textwidth}
		\begin{flushleft} \large
			\emph{Professor:}\\
			Elisabetta di Nitto\\
			\end{flushleft}
			\end{minipage}~
			\begin{minipage}{0.4\textwidth}
            
			\begin{flushright} \large
			\emph{Authors :} \\
			Andrea Biscontini - 901310\\
			Marco Gelli - 901470\\
			Alvise de'Faveri Tron - 920882\\
		\end{flushright}
        
	\end{minipage}\\[2 cm]
	
	
    
    
    
    
	
\end{titlepage}

%%%%%%%%%%%%%%%%%%%%%%%%%%%%%%%%%%%%%%%%%%%%%%%%%%%%%%%%%%%%%%%%%%%%%%%%%%%%%%%%%%%%%%%%%




\setcounter{page}{2}


%------------------------------------------------------------------------------------------------------------------------------------------------
\newpage
%\addcontentsline{toc}{section}{Table of Contents}
\tableofcontents
\newpage
%\addcontentsline{toc}{section}{List of Figures}
%\listoffigures
%\addcontentsline{toc}{section}{List of Tables}
%\listoftables

%------------------------------------------------------------------------------------------------------------------------------------------------
\clearpage
{\color{Blue}{\section{Introduction}}}
\label{sect:introduction}
prima dei sottocapitoli
\subsection{Purpose}
This document represent the Requirement Analysis and Specification Document of the Data4Help system. Here it's described the general purpose of the system, the functional and non-functional requirements that it must respect and the assumption through which we achieve all it's goal. This document is addressed to all the stakeholder of this system, which means final clients but also management, developer, testers and more.\\
The Data4Help system is composed by an user side smartphone Application and a web based query service for the third parties. The user-side App has the task to collect data from all the devices connected to the user's smartphone and send them to the TrackMe database. The web service is instead used by third parties to submit data request to TrackMe and, if they are successful, receive the most recent data collected on the proprietary databases.
AutomatedSOS is instead an user-side integration for the Data4Help service. Registered users' real time data are here monitored and, if there is any signal of possible health problems, local emergency numbers are called and an ambulance is called to intervene at the customer location.
Track4Run isanother ananotheranotheranotheranotherother system built upon Data4Help. Here users can become run organizers, enroll in scheduled runs and spectate live runs through a map with live GPS ranotheranotheranotherunners position.
\subsection{Scope}
\subsubsection{Description of the given problem}
As stated in the above section, Data4Help main goal is to collect user's data and made them available to third parties, all while guaranteeing the user privacy and consensus in personal data processing. To collect these data, the system needs to connect to users' smartwearables and download all the relevant produced data to TrackMe proprietary servers. Then this data are processed by TrackMe and whenever a request for data arrives from the third parties, if the request is successful, they're made available to them. Third parties could also desire to look for future changes in the data they requested, so an auxiliary subscription to new data is also made available at request time. Other than that, to simplify the data request procedure, this has been divided in two types of request: single user data request, that is forwarded directly to the individual that can accept or refuse it, and request for anonymized data of groups of individual, that is handled by TrackMe and it's always successful if there is the possibility to render the data anonym.
On top of the Data4Help system, that is used mainly by third party as a data retrieving service, there are AutomatedSOS and Track4Run. These service are instead to be used by an user of Data4Help. AutomatedSOS means is to help elderly or non-healty people to monitor their health status and intervene in the case of an emergency. In fact the goal of AutomatedSOS is to be very reactive (maximum 5 seconds) whenever a possible health problem is signaled and to immediately call emergency number and an ambulance for the location of the client.
Track4Run is instead a system designed for athletes and runners in which it's possible to organize, partecipate and spectate organized running competitions. Here any user can become the organizer of a run by creating one. The run creation procedure here is made really simple for the organizer, who needs only to insert the needed infos and select a route for the run on the map. When a run is created, every other user can enroll to it. To give an even better service, there is also the possibility to spectate a run on the App, which means follow every runner's position on a live GPS map.

The whole system ....
\subsubsection{Current System}
\subsubsection{Goals}
\begin{itemize}
\item \textbf{G1} Data4Help must be able to keep track of real time health status and position from registered users
\item \textbf{G2} Data4Help should allow third parties to gather information from a single user or from an anonymous group of users
\item \textbf{G3} Data4Help should allow third parties to subscribe to new data and receive them as soon as they're produced
\item \textbf{G4} AutomatedSOS should be able to identify an health emergency when the user data are below/exceeding a certain thresholds
\item \textbf{G5} AutomatedSOS must call an ambulance when it detects a health emergency
\item \textbf{G7} Track4Run allows a user to become an organizer of a run, so that he/she can create and manage a run
\item \textbf{G8} Track4Run allows a user to partecipate to an organized run
\item \textbf{G9} Track4Run allows a spectator to track the position of the partecipants of a run 
\end{itemize}
\subsection{Definitions, Acronyms, Abbreviations}
\subsubsection{Definitions}
\subsubsection{Acronyms}
\begin{itemize}
\item \textbf{D4H}: Data4Help
\item \textbf{ASOS}: AutomatedSOS
\item \textbf{T4R}: Track4Run
\item \textbf{RASD}: Requirement Analysis and Specification Document
\item \textbf{API}: Application Programming Interface
\item \textbf{GPS}: Global Positioning System

\end{itemize}
\subsubsection{Abbreviations}
\begin{itemize}
\item [Gn]: n-th goal
\item [Dn]: n-th domain assumption
\item [Rn]: n-th functional requirement
\end{itemize}
\subsection{Revision history}
quattro
\subsection{Reference Documents}
cinque
\subsection{Document Structure}
sei
%what you write here is a comment that is not shown in the final text

%------------------------------------------------------------------------------------------------------------------------------------------------
\clearpage
{\color{Blue}{\section{Overall Description}}}
\label{sect:overview}
\subsection{Product perspective}
The main goal of this system is to collect registered users' data, make them available and offer monitoring services to the end user.

The system is designed to be a set of three subsystems interacting with each other. In particular, Data4Help is the undelying subsystem that collects data from end users and makes it available to third party applications. 
The other two subsystems are built on top of Data4Help and provide to the user different services: AutomatedSOS offers an automated way of calling an ambulance in case of emergency, while Track4Run gives to the end users a platform to organize and participate to run events.

\begin{itemize}
\item \textbf{Data4Help:} is the main underlying subsystem, which .
- AutomatedSOS: is a subsystem built on top of Data4Help that provides the end user with a service to monitor his health status and 
\end{itemize}

\subsection{Product functions}

\subsubsection{User Data Acquisition}
\subsubsection{User Data Retrieval}
\subsubsection{Health Emergency Monitoring}
\subsubsection{Run Events Management}

\subsection{User characteristics}
\subsubsection{Actors}
\begin{itemize}
\item Data4Help User
\item Third Party
\item Data Source Application
\item Ambulance system
\item Run organizer
\item Run spectator
\item Runner
\item Sys Admin
\end{itemize}
\subsection{Assumptions, dependencies and constraints}
\subsubsection{Domain Assumptions}
\subsubsection{Software dependencies}
Software dependencies: Maps, ambulance API
\subsubsection{Hardware constraints}
Server(?), Smartphone/Smartwear/smartdevices, GPS, 4G, sensors

%------------------------------------------------------------------------------------------------------------------------------------------------
\clearpage
{\color{Blue}{\section{Specific Requirements}}}
\label{sect:requirements}
Organize this section according to the rules defined in the project description. 
\subsection{External Interface Requirements}
\subsubsection{User Interface}
\subsubsection{Hardware Interfaces}
\subsubsection{Software Interfaces}
\subsubsection{Communication Interfaces}
\subsection{Functional Requirements}
\subsection{Scenarios}
\subsubsection{Scenario 1}
John tells Frank about a new service called Data4Help that helps you keeping track of your health status. Frank, interested in checking his heart beat rate, downloads the app on his smartphone. He fires up  Data4Help and he register to the service by filling a form with his personal information, including the fiscal code and a password of his choice that will be used as credentials fo the log-in. 
After filling the form he clicks on the final checkbox to accept the personal data treatment policy.
Right after clicking on the "Submit" button, Frank receives a confirm alert that he has successfully registered to Data4Help.
\subsubsection{Scenario 2}
Mary, a Data4Help customer, received for her birthday the new Eppol iClock. After the initial setting of the device, she decides to download the Data4Help app for her smartwatch. Once installed, she logs-in with her account and right after that, the "Device Management" view of the application appears, asking Mary to manage her smartphone and her smartwatch. Mary taps on the iClock icon and by selecting the "Heart Rate" from a dropdown she assigns the tracking of that paramater to her smartwatch. Finally, with the same procedure, she assigns to her smartphone the tracking of the "Position" parameter.
To confirm the settings, she click on the "Ok" button and the app saves the changes and returns to the home page.
\subsubsection{Scenario 3}
Dr. Lemmings, after a medical examination, suggests to Steve, one of his patients with diabete, to periodically check the glucose levels observed by his medical device connected with the Data4Help application. 
When he got home, Steve starts the Data4Help app from his smartphone and he clicks on the "myData" tab in the homepage. So a nice view appears, containing all the information about his monitored health parameters with a lot of colorful diagrams. Steve filters out the displayed data by selecting the "Glucose Level" radio button on the top of the page and he changes the information granularity using a dropdown from "Year" to  "Month". 
\subsubsection{Scenario 4}
Michael Garcia, Boyern CEO, a big pharmaceutical company, heard about Data4Help service and decided in agreement with the Administrative Board of the company to introduce the software service in the company. To do that he visits  www.trackme.com/data4help web page and goes in the B2B solutions to register his company as a registered Third Party of the service. When the page shows up he sees that he must fill in a form to complete the procedure. 
Michael fills the form with all the requested information about the company, including the VAT-number, an e-mail and a password. After the filling procedure he clicks on the "Submit" button on the bottom and after some seconds he receives a mail with a link to complete the registration procedure.
\subsubsection{Scenario 5}
Brian, patient of the Lenox Hill Private Medical Center, was released  yesterday.  The clinic, registered to Data4Help service, decides to monitor his health status to see if the treatment he was under reached the desired results.
The doctor that was in charge of Brian logs-in to the Data4Help page of his clinic. Once logged in, he inserts Brian's fiscal code and selects from a checkbox "Heart Beat" and "Temperature" as the data to monitor. Finally he chooses from the dropdown  "3 months" for the observation period.
As soon as the doctor clicks on the "Send request" button on the bottom of the page, Brian receives a notification on his smartphone. Brian opens the application, taps on the "Requests" tab in the home page, selects the mentioned request and accepts it. After that Lenox Hill clinic receives notification e-mail saying that the patient accepted the treatment of his personal data. As soon as the doctor sees that e-mail, he goes in the "Sent Requests" section of the Data4Help personal page and after clicking on the Brian aswer all the requested data are displayed on the screen.
\subsubsection{Scenario 6}
Pfuzer, a big pharmaceutical company registered to Data4Help service, needs to gather the heart rate data of all the italian young people under 30 years old for a market analysis aimed at evaluating the production of a new drug against heart disease. For this reason, the marketing manager of the company Todd Chavez goes to the Pfuzer personal Data4Help page and once logged  in he clicks on the "Aggregated requests" tab in the home page. Once in the page he types in the textbox "Italy" as location, and selects the age range "<= 30" using a slider. Then he clicks on "Send request" on the bottom of the page and an warning message is immediately displayed on screen saying that the request cannot be satisfied. Todd, decides to untighten the search parameters and after changing them the request is satisfied and the requested data is immediately displayed on screen. 
\subsection{Performance Requirements}
\subsection{Design Constraints}
\subsubsection{Standard compliance}
\subsubsection{Hardware limitations}
\subsubsection{Any other constraint}
\subsection{Software System Attributes}
\subsubsection{Reliability}
\subsubsection{Availability}
\subsubsection{Security}
\subsubsection{Maintainability}
\subsubsection{Portability}




%------------------------------------------------------------------------------------------------------------------------------------------------
\clearpage


\newpage
\subsection{Performance Requirements}
In general, the main performance requirements for our system concern the time needed to process data and fulfill data requests.

In particular, the system should meet the following strong requirement:

\begin{itemize}
	\item \textbf{AutomatedSOS Emergency Response}
	The system must call an ambulance within \textbf{5 seconds} from when a parameter exceeds a threshold. This time shall take into account the time needed for the AutomatedSOS monitoring system to process new incoming data and detect the emergency, the Data4Help system to send the new data to AutomatedSOS and the average delay introduced from the communication interfaces.
\end{itemize}

The following performance requirements are additionally introduced, although with a minor importance compared to the previous one. They should anyway be met from the software, in order for it to be usable and efficient:

\begin{itemize}
	\item \textbf{Data4Help Single Data Request Response}
	After a single data request is accepted, the system should send the requested data to the third party within 3 seconds. 
	
	\item \textbf{Data4Help Data Subscription Delay }
	When a new data is received from the system, the subscribed third parties should be notified of the new data within 2 seconds.
\end{itemize}

\subsection{Design Constraints}
\subsubsection{Regulatory policies}
\begin{itemize}
	\item \textbf{GDPR}: In order to protect the user's privacy and perform a correct data treatment, the system must be developed in compliance with the latest GDPR regulations, giving the maximum attention to the protection and anonymization of the user's data.
\end{itemize}
\subsubsection{Hardware limitations}
The only hardware limitations of our system is present in the devices on which AutomatedSOS and Track4Run applications will be installed. For this reason, the development of the software to be will have to reasonably take in account some common limitations of these devices, such as power consumption and memory size.


\subsection{Software System Attributes}
\subsubsection{Reliability}

The system must be available 24/7. Since this requirement is quite demanding, small service breaks will be tolerated. To boost reliability, a RAID architecture that combines multiple disk drives to have data redundancy is suggested. Using a RAID controller we could also expand disks' capacity or substitute them without the risk of losing data or suspending the service for too long. Preventive maintenance is also a good idea to avoid downtimes.

\subsubsection{Availability}

To guarantee a 3-nines (99.9\%) availability degree, a system of redundant servers could be considered. In this way we don't have a single point of failure and if one server fails, another one will be ready to substitute it.

\subsubsection{Security}

Security is a main issue.  At first, to access the functionalities offered by our platform, users have to complete a login phase providing their credentials: fiscal code and password in case of a single user or VAT code and electronic signature for the third parties. These sensitive information should be confidentially stored and encrypted using a proper hash function.

Secondly, the system  continuously collects data on health conditions from the users. Typically this kind of data are also considered sensitive and should be kept secret. Finally, also communications between users and our platform is very important, for this reason the system could be accessed by the customers using secure connection protocols like HTTPS to avoid Man In The Middle attacks. Only the system administrator who is responsible of the Web and Application Server configuration can decide if it is the case to implement them or not.

\subsubsection{Maintainability}

In order to make our software the most maintainable as possible, we will adopt a modular design for our code. In this different components of the system can be modified and additional ones can be introduced in the system with whenever is needed and with a minimal impact on the rest of the system.

Moreover, we will make sure to adopt standard design patterns and coding best practices, so that our code can be easily understood and modified by any future developer.

Finally, the code should be completely and properly documented in order to facilitate the understanding of it. 

\subsubsection{Portability}

During the system installation and setup, we have to consider different factors:

\begin{itemize}
	\item Ease of installation on the central server
	\item Scalability of the platform considering future adjustments without losing the already collected data
	\item Portability of data between different machines, in order to move the system on several more powerful machines in case of necessity
	up 24/7
\end{itemize}




{\color{Blue}{\section{Formal Analysis Using Alloy}}}
\label{sect:alloy}
In this final section there is the alloy model, the world generated and the proof of consistency.
The model mainly describes the relationships between the various Data4Help entities. It's mainly shown how:
\begin{itemize}
\item It's impossible that multiple data sources of the same user tracks the same parameter.
\item Requests with filters (not targeted at a single user's data, but at a group of users' data defined by that filters and anonymized) are accepted by the system only if the filters cover a wide enough group to guarantee privacy.
\item Each user of AutomatedSOS is monitored by the system and, if hearth beat is lower or higher than the threshold range, the user is signaled to be in danger.
\item Each user that want to enroll in a run is monitored on his position by Track4Run
\item Each Spectator of Track4Run can only watch ongoing run and only if in that run are involved runners which positions are tracked
\end{itemize}
Moreover the world generated has 2 projected signatures: DataSources and Int. The consistency between the threshold values is nonetheless valid, even if omitted for readability. The same reasoning applies for DataSources, their are omitted to keep the image readable.\\

\lstinputlisting[language=alloy]{../Alloy/alloyReview.als}
\subsection{World Generated}
\centering
\includegraphics[angle=90, width =\textwidth, height=\textheight, keepaspectratio]{../Alloy/worldGenerated.jpg}\\
\subsection{Proof of consistency}
{
\centering
\includegraphics[width = 0.9\textwidth]{../Alloy/proofOfConcept.jpg}\\
}


%------------------------------------------------------------------------------------------------------------------------------------------------
\clearpage
{\color{Blue}{\section{Effort Spent}}}
\label{sect:effort}
\subsection{Andrea Biscontini}
\begin{itemize}
	\item General brainstorming : 5h
	\item Requirements bainstorming: 7h
	\item Alloy model : 10h
	\item Final review : 15h
	\item Alloy review : 8h
	\item Scenarios and Use Case review : 1h
\end{itemize}
\subsection{Alvise de' Faveri Tron}
\begin{itemize}
	\item General brainstorming : 5h
	\item Requirements bainstorming: 7h
	\item UML models : 8h
	\item Final review : 15h
	\item Alloy review : 2h
\end{itemize}
\subsection{Marco Gelli}
\begin{itemize}
	\item General brainstorming : 5h
	\item Requirements bainstorming: 7h
	\item UI mockups : 9h
	\item Final review : 15h
	\item Scenarios and Use Case review : 5h
\end{itemize}

%------------------------------------------------------------------------------------------------------------------------------------------------
\clearpage
\graphicspath{ {../Mockups/} }

% Data4Help Mockups

% User

\begin{figure}
  \includegraphics[width=\linewidth]{Data4Help/userRegistration.png}
  \caption{Data4Help - User Registration}
  \label{Data4Help - User Registration}
\end{figure}

\begin{figure}
  \includegraphics[width=\linewidth]{Data4Help/userLogin.png}
  \caption{Data4Help - User Login}
  \label{Data4Help - User Login}
\end{figure}

\begin{figure}
  \includegraphics[width=\linewidth]{Data4Help/userDataSources.png}
  \caption{Data4Help - User Data Sources Management}
  \label{Data4Help - User Data Sources Management}
\end{figure}

\begin{figure}
  \includegraphics[width=\linewidth]{Data4Help/userThirdPartiesManagement.png}
  \caption{Data4Help - User Third Parties Management}
  \label{Data4Help - User Third Parties Management}
\end{figure}

% Third Parties

\begin{figure}
  \includegraphics[width=0.8\linewidth]{Data4Help/thirdPartyRegistration@2x.png}
  \caption{Data4Help - Third Party Registration}
  \label{Data4Help - Third Party Registration}
\end{figure}

\begin{figure}
  \includegraphics[width=\linewidth]{Data4Help/thirdPartyLogin@2x.png}
  \caption{Data4Help - Third Party Login}
  \label{Data4Help - Third Party Login}
\end{figure}

\begin{figure}
  \includegraphics[width=\linewidth]{Data4Help/thirdPartyNewSingleRequest@2x.png}
  \caption{Data4Help - Third Party New Single Request}
  \label{Data4Help - Third Party New Single Request}
\end{figure}

\begin{figure}
  \includegraphics[width=\linewidth]{Data4Help/thirdPartyNewGroupRequest@2x.png}
  \caption{Data4Help - Third Party New Group Request}
  \label{Data4Help - Third Party New Group Request}
\end{figure} 

\begin{figure}
  \includegraphics[width=\linewidth]{Data4Help/thirdPartyAcceptedRequests@2x.png}
  \caption{Data4Help - Third Party Accepted Requests}
  \label{Data4Help - Third Party Accepted Requests}
\end{figure}

% AutomatedSOS Mockups

\begin{figure}[!ht]
  \centering
  \begin{subfigure}[b]{0.4\linewidth}
    \includegraphics[width=\linewidth]{AutomatedSOS/login.png}
    \caption{Login}
  \end{subfigure}\hfill
  \begin{subfigure}[b]{0.4\linewidth}
    \includegraphics[width=\linewidth]{AutomatedSOS/thresholdSettings.png}
    \caption{Threshold Settings}
  \end{subfigure}
  \par\bigskip
  \begin{subfigure}[b]{0.4\linewidth}
    \includegraphics[width=\linewidth]{AutomatedSOS/myData.png}
    \caption{myData}
  \end{subfigure}\hfill
  \begin{subfigure}[b]{0.4\linewidth}
    \includegraphics[width=\linewidth]{AutomatedSOS/emergencyAlert.png}
    \caption{Emergency Alert}
  \end{subfigure}
  \caption{AutomatedSOS Mockups}
\end{figure}

% Track4Run Mockups

\begin{figure}[!ht]
  \centering
  \begin{subfigure}[b]{0.4\linewidth}
    \includegraphics[width=\linewidth]{Track4Run/login.png}
    \caption{Login}
  \end{subfigure}\hfill
  \begin{subfigure}[b]{0.4\linewidth}
    \includegraphics[width=\linewidth]{Track4Run/myRuns.png}
    \caption{myRuns}
  \end{subfigure}
  \par\bigskip
  \begin{subfigure}[b]{0.3\linewidth}
    \includegraphics[width=\linewidth]{Track4Run/search.png}
    \caption{Search}
  \end{subfigure}\hfill
  \begin{subfigure}[b]{0.3\linewidth}
    \includegraphics[width=\linewidth]{Track4Run/infoMap.png}
    \caption{Run Details}
  \end{subfigure}\hfill
  \begin{subfigure}[b]{0.3\linewidth}
    \includegraphics[width=\linewidth]{Track4Run/infoMapMarked.png}
    \caption{Live Run Details}
  \end{subfigure}
  \caption{Track4Run Mockups}
\end{figure}
{\color{Blue}{\section{Appendix A}}}
\label{sect:Appendix A}



\end{document}