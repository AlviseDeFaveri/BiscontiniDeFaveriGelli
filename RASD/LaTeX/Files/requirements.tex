Organize this section according to the rules defined in the project description. 
\subsection{External Interface Requirements}
\subsubsection{User Interface}
\subsubsection{Hardware Interfaces}
\subsubsection{Software Interfaces}
\subsubsection{Communication Interfaces}
\subsection{Functional Requirements}
\subsection{Scenarios}
\subsubsection{Scenario 1}
While talking, John tells his friend Matt about a new service called Data4Help that helps you keeping track of your health status. Frank, interested in checking his heart rate, downloads the app on his smartphone and fires it up. The app asks him to register to the service by filling a form with his personal information, including his fiscal code and a password of his choice that will be used as credentials for the login. 
After filling the form he clicks on the final checkbox to accept the personal data treatment policy.
Right after clicking on the "Submit" button, Frank receives a notification saying he successfully registered to Data4Help.
\subsubsection{Scenario 2}
Mary, a Data4Help customer, received for her birthday the new Eppol iClock. After the initial setting of the device, she decides to download the Data4Help application for her smartwatch. Once installed, she logs in with her accountand right after that, the "Device Management" view of the application appears. Mary taps on the iClock icon and by selecting "Heart Rate" from a dropdown she assigns the tracking of that paramater to her smartwatch. With the same procedure, she assigns to her smartphone the tracking of the "Position" parameter.
Finally, she clicks on the "Confirm" button and the app saves the settings and returns back to the home page.
\subsubsection{Scenario 3}
Steve, a Data4Help customer with diabete, needs to periodically check the glucose levels observed by his medical device connected with the Data4Help application on his smartphone. For doing this, he starts the Data4Help application and when the homepage shows up he  clicks on the "myData" tab. A nice view appears, containing all the information about his monitored health parameters with a lot of colorful diagrams. Steve f ilters out the displayed content by selecting the "Glucose Level" radio button on the top of the page and he changes the information granularity from "Week" to "Day" using a dropdown.  
\subsubsection{Scenario 4}
Michael Garcia, Boyer Pharmaceutical CEO, heard about the Data4Help service and in agreement with the Administrative Board, he decided to introduce it in the company. He visits  www.trackme.com/data4help and clicks on the "Business Solutions" section in order to register his company as a registered third party of the service. Tp do this, Michael fills in the registration form with all the requested information about the company, including the VAT-number, an e-mail and a password. After that he clicks on the "Submit" button and he immediately receives an e-mail with a link to complete the registration procedure.
\subsubsection{Scenario 5}
Brian, patient of the Lenox Private Medical Center, was released  yesterday. The clinic, registered to the Data4Help service, decides to monitor his health status to see if the treatment he was under reached the desired results.
The doctor that was in charge of Brian logs in to the Data4Help reserved page of the clinic. Once logged in, he inserts Brian's fiscal code and selects from a checkbox "Heart Beat" and "Temperature" as the data to monitor. Finally he chooses from a dropdown "3 months" as the observation period.
As soon as the doctor clicks on the "Send Request" button on the bottom of the page, Brian receives a notification on his smartphone. He launches the application and when the home page shows up he taps on the "Incoming Requests" tab. Here he sees the request coming from Lenox Private Medical Center. He briefly reads the description and he clicks on the "Accept" button. A confirmation e-mail, saying the patient accepted the treatment of his personal data, is immediately sent to Lenox Private Medical Center.
The doctor sees the e-mail and goes in the "Sent Requests" section of the Data4Help personal page of the company. After clicking on Brian's answer, all the requested data are displayed on screen.
\subsubsection{Scenario 6}
Pfuzer, a big pharmaceutical company registered to Data4Help service, needs to gather the heart rate data of all the italian young people under 30 years old for a market analysis aimed at evaluating the production of a new drug against heart disease. For this reason Todd Chavez, the marketing manager of the company, goes to the Pfuzer personal Data4Help page and once logged  in he clicks on the "Aggregated requests" tab in the home page. Once in the page he types in the textbox "Italy" as location, and selects the age range "<= 30" using a slider. Then he clicks on "Send request" on the bottom of the page and a warning message is immediately displayed on screen saying that the request cannot be satisfied. For this reason, Todd decides to untighten the search parameters. After changing them the constraints are satisfied and the requested data is immediately displayed on screen.
\subsubsection{Scenario 7}
\subsubsection{Scenario 8}
\subsubsection{Scenario 9}
\subsubsection{Scenario 10}
\subsection{Performance Requirements}
\subsection{Design Constraints}
\subsubsection{Standard compliance}
\subsubsection{Hardware limitations}
\subsubsection{Any other constraint}
\subsection{Software System Attributes}
\subsubsection{Reliability}
\subsubsection{Availability}
\subsubsection{Security}
\subsubsection{Maintainability}
\subsubsection{Portability}
