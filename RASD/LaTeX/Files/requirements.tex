Organize this section according to the rules defined in the project description. 
\subsection{External Interface Requirements}
\subsubsection{User Interface}
mockup
\subsubsection{Hardware Interfaces}
none
\subsubsection{Software Interfaces}
Data acquisition and data retrieval
\subsubsection{Communication Interfaces}
REST API, TCP/IP, HTTPS
\subsection{Functional Requirements}
\textbf{GG} Data4Help allows the user to examine all his/her collected data

\textbf{[G1] Data4Help must be able to keep track of real time health status and position from registered users' devices}
\begin{itemize}
\item \textbf{R1} A user can register to TrackMe's services providing a Fiscal code and a password of their choice
\item \textbf{D1} Fiscal code uniquely identifies a user of the system
\item \textbf{R2} After registration, a user can log in by using his/her credentials
\item \textbf{R3} The system acquires user data only after he/she accepted the data acquisition policy
\item \textbf{D2} Every user owns at least one device capable of retrieving correct real-time health parameters and location
\item \textbf{D3} User devices must grant to the system access to the requested data
\item \textbf{D4} User devices must be continuously connected to the internet
\end{itemize}

\textbf{G2} Data4Help should allow third parties to gather information from a single user or from an anonymous group of users
\begin{itemize}
\item \textbf{R4} Third parties can register to Data4Help by providing a VAT code
\item \textbf{D5} VAT code uniquely identifies a third party in the system
\item \textbf{R5} The system provides to the registered third parties unique access tokens to use the services
\item \textbf{R6} The system grants access to a single user data only after his/her confirmation
\item \textbf{D6} Third parties know the monitored user Fiscal Code
\item \textit{R7 The system is capable of merging multiple user data and anonymize it}
\item \textbf{R8} The system shall grant access to merged data only if the number	of individuals	whose data satisfy the request is higher than	 1000.	
 	
  

\end{itemize}


\textbf{G3} Data4Help should allow third parties to subscribe to new data and receive them as soon as they're produced
\begin{itemize}
\item \textbf{G4} AutomatedSOS must use data coming from Data4Help to monitor the health condition of its users
\item \textbf{G5} AutomatedSOS should be able to identify an health emergency when the user data are below/exceeding a certain thresholds
\item \textbf{G6} AutomatedSOS must call an ambulance when it detects a health emergency
\item \textbf{G7} Track4Run allows a user to become an organizer of a run, so that he/she can create and manage a run
\item \textbf{G8} Track4Run allows a user to partecipate to an organized run
\item \textbf{G9} Track4Run allows a spectator to track the position of the partecipants of the run 

\end{itemize}


\subsection{Scenarios}
\subsubsection{Scenario 1}
While talking, John tells his friend Matt about a new service called Data4Help that helps you keeping track of your health status. Frank, interested in checking his heart rate, downloads the app on his smartphone and fires it up. The app asks him to register to the service by filling a form with his personal information, including his fiscal code and a password of his choice that will be used as credentials for the login. 
After filling the form he clicks on the final checkbox to accept the personal data treatment policy.
Right after clicking on the "Submit" button, Frank receives a notification saying he successfully registered to Data4Help.
\subsubsection{Scenario 2}
Mary, a Data4Help customer, received for her birthday the new Eppol iClock. After the initial setting of the device, she decides to download the Data4Help application for her smartwatch. Once installed, she logs in with her accountand right after that, the "Device Management"
%secondo me meglio un home page qui
 view of the application appears. Mary taps on the iClock icon and by selecting "Heart Rate" from a dropdown she assigns the tracking of that paramater to her smartwatch. With the same procedure, she assigns to her smartphone the tracking of the "Position" parameter.
Finally, she clicks on the "Confirm" button and the app saves the settings and returns back to the home page.
\subsubsection{Scenario 3}
Steve, a Data4Help customer with diabete, needs to periodically check the glucose levels observed by his medical device connected with the Data4Help application on his smartphone. For doing this, he starts the Data4Help application and when the homepage shows up he  clicks on the "myData" tab. A nice view appears, containing all the information about his monitored health parameters with a lot of colorful diagrams. Steve filters out the displayed content by selecting the "Glucose Level" radio button on the top of the page and he changes the information granularity from "Week" to "Day" using a dropdown.  
\subsubsection{Scenario 4}
Michael Garcia, Boyer Pharmaceutical CEO, heard about the Data4Help service and in agreement with the Administrative Board, he decided to introduce it in the company. He visits  www.trackme.com/data4help and clicks on the "Business Solutions" section in order to register his company as a registered third party of the service. To do this, Michael fills in the registration form with all the requested information about the company, including the VAT-number, an e-mail and a password. After that he clicks on the "Submit" button and he immediately receives an e-mail with a link to complete the registration procedure.
\subsubsection{Scenario 5}
Brian, patient of the Lenox Private Medical Center, was released  yesterday. The clinic, registered to the Data4Help service, decides to monitor his health status to see if the treatment he was under reached the desired results.
The doctor that was in charge of Brian logs in to the Data4Help reserved page of the clinic. Once logged in, he inserts Brian's fiscal code and selects from a checkbox "Heart Beat" and "Temperature" as the data to monitor. Finally he chooses from a dropdown "3 months" as the observation period.
As soon as the doctor clicks on the "Send Request" button on the bottom of the page, Brian receives a notification on his smartphone. He launches the application and when the home page shows up he taps on the "Incoming Requests" tab. Here he sees the request coming from Lenox Private Medical Center. He briefly reads the description and he clicks on the "Accept" button. A confirmation e-mail, saying the patient accepted the treatment of his personal data, is immediately sent to Lenox Private Medical Center.
The doctor sees the e-mail and goes in the "Sent Requests" section of the Data4Help personal page of the company. After clicking on Brian's answer, all the requested data are displayed on screen.
\subsubsection{Scenario 6}
Pfuzer, a big pharmaceutical company registered to Data4Help service, needs to gather the heart rate data of all the italian young people under 30 years old for a market analysis aimed at evaluating the production of a new drug against heart disease. For this reason Todd Chavez, the marketing manager of the company, goes to the Pfuzer personal Data4Help page and once logged  in he clicks on the "Aggregated requests" tab in the home page. Once in the page he types in the textbox "Italy" as location, and selects the age range "<= 30" using a slider. Then he clicks on "Send request" on the bottom of the page and a warning message is immediately displayed on screen saying that the request cannot be satisfied. For this reason, Todd decides to untighten the search parameters. After changing them the constraints are satisfied and the requested data is immediately displayed on screen.
\subsubsection{Scenario 7}
Andrew decided to buy the new iClock as a gift for the birthday of his grandfather Leonard, 93 years old, who suffers from severe heart problems and lives alone. Since Leonard would like to be autonomous, his new device with AuotmatedSOS service active on it, grants him a safer life.

One night Leonard wakes up with severe chest pains. The iClock immediately detects the "heart rate" parameter exceeded the maximum threshold and shows a noisy emergency alert, saying that an ambulance will be called if the user doesn't abort in the following minute. Leonard is really sick and is not able to abort the operation. In few minutes an ambulance arrives and Leonard is immediately rescued.
\subsubsection{Scenario 8}
William suffers from epilepsy. The attacks are not very frequent but they are so strong that a few months ago he fell and he got a nasty head injury. While looking for an automated solution to check on his conditions, he hears about AutomatedSOS and decides to activate the service provided by Data4Help.

At first he sets the threshold of the tracked parameters with the help of his doctor and he adds the contacts of his parents. When the wristwatch detects repetitive shaking motion, it automatically sends the user’s bluetooth-connected phone text and call alerts to the designated recipients. Within seconds, family members receive these alerts, which include the date, time and GPS location of the event.
\subsubsection{Scenario 9}
Virgin Active is organizing a run for all his customers. Luke, company's event manager, is thinking about using the Track4Run Service to accomplish this task. He logs into the company dedicated web page and he clicks on the Track4Run tab and then on "Create Run".

At this point a configuration frame shows up. Luke starts by filling out the "Basic info" section of the form: he adds the event name, an image, the location and the starting and ending time. Then he goes on by filling out the "Details" section in which he provides a short description of the event and the maximum number of participants. Finally, by clicking on a map, he marks all the checkpoints of the run.

After a quick check, Luke clicks on the "Create" button and a confirmation alert appears saying that the event has been successfully published in the news feed.
\subsubsection{Scenario 10}
Chris, runner and Data4Help customer, is looking for a run to join. By looking at the  news feed of Track4Run on his smartphone he sees the event created by Virgin Active. He taps on the event name and the full description of the event shows up. After reading the description and all the details he decides to join the run, hence he clicks on the "Participate" button. A confirmation alert shows up, saying that the event has been added to the attending events.
\subsubsection{Scenario 11}
Katy, Chris' wife and Data4Help customer,  wants to go cheer his husband at the run. The day of the run she launches the Data4Help application and in the Track4Run section she taps on the "Ongoing Runs" button. Katy selects the desired run and a nice map with a the GPS position of all the athletes opens up. In order to see the details of the husband she types in a textbox her husband name and press the "Filter" button. After that on the map is shown only the position of his husband followed by the his timing performances, the heart rate and the calories burnt.
 
\subsection{Use Case}
\begin{table}[]
\begin{tabular}{|l|p{12cm}|}
\hline
Name             & User registration \\ \hline
Actor            & User \\ \hline
Entry conditions & User enters in the Data4Help web page \\ \hline
Events flow      & \begin{enumerate}
\item Click on the "Sign Up" button
\item Fill the registration form and the account credentials
\item Accept the terms and conditions by clicking on the checkbox 
\item Click on the "Sign Up" button
\item The system elaborate the registration and send back a notification
\end{enumerate} 
\\ \hline
Exit conditions  & Registration is successful and the user is informed via notification \\ \hline
Exceptions       & \begin{enumerate}
\item The user is already registered
\item There is some invalid data in the form
\item The email is already used
\item Terms and conditions haven't been checked
\end{enumerate} All the exceptions take the user back to the registration procedure \\ \hline
\end{tabular}
\end{table}

\begin{table}[]
\begin{tabular}{|l|p{12cm}|}
\hline
Name             & User Login \\ \hline
Actor            & User \\ \hline
Entry conditions & User already registered and enters Data4Help web page\\ \hline
Events flow      & \begin{enumerate}
\item Click on the "Log in" button
\item The user enter his/her credentials
\item Click on the "Enter" button
\item The log in was successful and the user is redirected to the home page of the App
\end{enumerate} \\ \hline
Exit conditions  & The log in is successful and the user is redirected to home page \\ \hline
Exceptions       & \begin{enumerate}
\item Credentials aren't valid
\end{enumerate} The exceptions are notified to the user and the Login procedure restart \\ \hline
\end{tabular}
\end{table}

\begin{table}[]
\begin{tabular}{|l|p{12cm}|}
\hline
Name             & Add Data Source \\ \hline
Actor            & User \\ \hline
Entry conditions & User synchronizes his device or application with Data4Help \\ \hline
Events flow      & \begin{enumerate}
\item A confirmation email is sent to the user
\item The user reads the email and clicks on the provided confirmation link
\item The data source is added to user account 
\end{enumerate} \\ \hline

Exit conditions  & The user confirms the synchronization
Exceptions       & \begin{enumerate}
\item The user didn't request the synchronization and doesn't click on the confirmation link
\end{enumerate} After 24 hours the request is made void and a notification email is sent to the user\\ \hline
\end{tabular}
\end{table}

\begin{table}[]
\begin{tabular}{|l|p{12cm}|}
\hline
Name             & Data Sources Management \\ \hline
Actor            & User \\ \hline
Entry conditions & User enter in the "Data Sources" section of the web site \\ \hline
Events flow      & \begin{enumerate}
\item The list of data sources connected to the account is shown
\item The user select which source to configure
\item For that source, the list of all the possible parameter that it can track is shown
\item The user turns On/Off the tracking of each parameter
\item Clicks on the "Save" button
\end{enumerate} \\ \hline

Exit conditions  & The user have set his/her preference and saves them \\ \hline
Exceptions       & \begin{enumerate}
\item the parameter is already tracked from a more reliable source (maybe??)
\end{enumerate} \\ \hline
\end{tabular}
\end{table}

\begin{table}[]
\begin{tabular}{|l|p{12cm}|}
\hline
Name             & Third Party Registration \\ \hline
Actor            & Third Party \\ \hline
Entry conditions & The third party clicks on "Business" on www.trackme.com/data4help \\ \hline
Events flow      & \begin{enumerate}
\item Clicks on the "Sign Up" button
\item Fills the form with information regarding the company
\item Clicks on the "Sign Up" button
\end{enumerate} \\ \hline
Exit conditions  & Registration is successful and the user is informed via notification \\ \hline
Exceptions       & \begin{enumerate}
\item Company already registered
\item Electronic Signature not valid
\end{enumerate} All the exceptions are notified and the procedure goes back to registration \\ \hline
\end{tabular}
\end{table}

\begin{table}[]
\begin{tabular}{|l|p{12cm}|}
\hline
Name             & Third Party Login \\ \hline
Actor            & Third Party \\ \hline
Entry conditions & The third party goes on www.trackme.com/data4help and click "Login" \\ \hline
Events flow      & \begin{enumerate}
\item Click on the "Login" button
\item Enter the company credentials
\item Click on the "Login" button
\end{enumerate} \\ \hline
Exit conditions  & Login is successful and the client is redirected to its reserved page  \\ \hline
Exceptions       & \begin{enumerate}
\item Credentials are not valid
\end{enumerate} The exceptions are notified to the client and the Login procedure restart  \\ \hline
\end{tabular}
\end{table}

\begin{table}[]
\begin{tabular}{|l|p{12cm}|}
\hline
Name             & One Shot Single Data Request \\ \hline
Actor            & Third Party, User \\ \hline
Entry conditions & A third party client is logged in and goes in the "Requests" section\\ \hline
Events flow      & \begin{enumerate}
\item Clicks on "New Request" button
\item Selects "Single Request" radio button
\item Selects the "One Shot" radio button
\item Inserts the fiscal code of the recipient and the request name in text fields
\item Inserts a brief description of the request by filling a text area
\item Checks which parameters to monitor from a checklist
\item Clicks on the "Send" button
\item The request is notified to the user via email
\item The user checks his email and clicks on the confirmation link
\item The result is notified via e-mail to the third party
\item If the user accepted, the information are made available under the "Accepted" section of the "Request" Data4Help personal page of the company
\end{enumerate} \\ \hline
Exit conditions  & The user has responded to the request and, if successful, the data are made available to the third company \\ \hline
Exceptions \begin{enumerate}
\item No user found that correspond to the search
\item The user refuses the request
\end{enumerate} This exception is notified to the third party and the request ends\\  \hline
\end{tabular}
\end{table}

\begin{table}[]
\begin{tabular}{|l|p{12cm}|}
\hline
Name             & Subscription Single Data Request \\ \hline
Actor            & Third Party, User \\ \hline
Entry conditions & A third party client is logged in and goes in the "Requests" section\\ \hline
Events flow      & \begin{enumerate}
\item Clicks on "New Request" button
\item Selects "Single Request" radio button
\item Selects the "Subscription Period" from a dropdown
\item Inserts the fiscal code of the recipient and the request name in text fields
\item Inserts a brief description of the request by filling a text area
\item Checks which parameters to monitor from a checklist
\item Clicks on the "Send" button
\item The request is notified to the user via email
\item The user checks his email and clicks on the confirmation link
\item The result is notified via e-mail to the third party
\item If the user accepted, the information are made available under the "Accepted" section of the "Request" Data4Help personal page of the company
\item Every time that the requested data changes in the Data4Help database, the data available to the third party are updated
\end{enumerate} \\ \hline
Exit conditions  & The Subscription Period ends \\ \hline
Exceptions \begin{enumerate}
\item No user found that correspond to the search
\item The user refuses the request
\end{enumerate} This exception is notified to the third party and the request ends\\  \hline
\end{tabular}
\end{table}

\begin{table}[]
\begin{tabular}{|l|p{12cm}|}
\hline
Name             & One Shot Group Data Request \\ \hline
Actor            & Third Party \\ \hline
Entry conditions & A third party client is logged in and goes in the "Requests" section\\ \hline
Events flow      & \begin{enumerate}
\item Clicks on "New Request" button
\item Selects "Group Request" radio button
\item Selects the "One Shot" radio button
\item Inserts the request name in a text field
\item Specificy the search area by filling a short form
\item Inserts a brief description of the request by filling a text area
\item Checks which parameters to monitor from a checklist
\item Clicks on the "Send" button
\item The result is notified via e-mail to the third party
\item If successfull, the information are made available under the "Accepted" section of the "Request" Data4Help personal page of the company
\end{enumerate} \\ \hline
Exit conditions  & The request was successful and the data are made available to the third company \\ \hline
Exceptions       
\begin{enumerate}
\item Request is rejected due to lack of anonymization (less than 1000 users)
\end{enumerate} The exception notifies the third party on reasons of the rejection and returns to the Data Request page\\ \hline
\end{tabular}
\end{table}

\begin{table}[]
\begin{tabular}{|l|p{12cm}|}
\hline
Name             & Subscription Group Data Request \\ \hline
Actor            & Third Party \\ \hline
Entry conditions & A third party client is logged in and goes in the "Requests" section\\ \hline
Events flow      & \begin{enumerate}
\item Clicks on "New Request" button
\item Selects "Group Request" radio button
\item Selects the "Subscription Period" from a dropdown
\item Inserts the request name in a text field
\item Specificy the search area by filling a short form
\item Inserts a brief description of the request by filling a text area
\item Checks which parameters to monitor from a checklist
\item Clicks on the "Send" button
\item The result is notified via e-mail to the third party
\item If successfull, the information are made available under the "Accepted" section of the "Request" Data4Help personal page of the companyù
\item Every time that the requested data changes in the Data4Help database, the data available to the third party are updated
\end{enumerate} \\ \hline
Exit conditions  & The Subscription Period ends \\ \hline
Exceptions       
\begin{enumerate}
\item Request is rejected due to lack of anonymization (less than 1000 users)
\end{enumerate} The exception notifies the third party on reasons of the rejection and returns to the Data Request page\\ \hline
\end{tabular}
\end{table}

\begin{table}[]
\begin{tabular}{|l|p{12cm}|}
\hline
Name             & AutomatedSOS Login \\ \hline
Actor            & User \\ \hline
Entry conditions & The user has previously installed the AutomatedSOS app on his smartphone \\ \hline
Events flow      & \begin{enumerate}
\item Clicks on the "Login" button
\item Enters his account credentials
\item Click on the "Login" button
\end{enumerate} \\ \hline
Exit conditions  & Login is successful\\ \hline
Exceptions       & \begin{enumerate}
\item Credentials are not valid
\end{enumerate} The exceptions are notified to the client and the Login procedure restart  \\ \hline
\end{tabular}
\end{table}

\begin{table}[]
\begin{tabular}{|l|p{12cm}|}
\hline
Name             & Monitoring Service \\ \hline
Actor            & AutomatedSOS, User, Ambulance \\ \hline
Entry conditions & The user previously installed AutomatedSOS application and logged with his Data4Help account\\ \hline
Events flow      & \begin{enumerate}
\item The Monitoring Service continuously checks the available real time data
\item Checks for threshold constraints
\item Whenever a parameter is found below or above its thresholds the user is given an emergency status
\item On the user's smartdevice, an emergency status screen with a countdown to call an ambulance shows up
\item An API call for an ambulance in the users' with emergency status location is made
\end{enumerate} \\ \hline
Exit conditions  & User unsubscribe from Data4Help or unistall AutomatedSOS application \\ \hline
Exceptions       &  \begin{itemize}
\item Lost physical device for data acquisition
\end{itemize} This exception notifies the client and pause the Monitoring System until the problem is fixed\begin{itemize}
\item User clicks on the "Abort" in the emergency status screen
\end{itemize} This exception revokes the emergency status of the user \\ \hline
\end{tabular}
\end{table}

\begin{table}[]
\begin{tabular}{|l|p{12cm}|}
\hline
Name             & Threshold Settings \\ \hline
Actor            & User \\ \hline
Entry conditions & User enter in the "Threshold Settings" tab of the AutomatedSOS application \\ \hline
Events flow      & \begin{enumerate}
\item The system shows tracked parameters, default values and saved values for each of them
\item For every parametery the user can change the actual value of the thresholds by chosing from a dropdown
\item The user saves the changes
\end{enumerate} \\ \hline
Exit conditions  & The user saves and exits \\ \hline
Exceptions       &  \\ \hline
\end{tabular}
\end{table}

\begin{table}[]
\begin{tabular}{|l|p{12cm}|}
\hline
Name             & MyData \\ \hline
Actor            & User \\ \hline
Entry conditions & User enter in the "MyData" tab of the AutomatedSOS application \\ \hline
Events flow      & \begin{enumerate}
\item The system shows all the user gathered info
\item The user can select from a dropdown the observation period
\item Whenever a filer is changed the app respond with the filtered information
\end{enumerate} \\ \hline
Exit conditions  & The user exits the "myData" tab \\ \hline
Exceptions       & \begin{enumerate}
\item the system haven't gathered any info yet
\end{enumerate} The exceptions are notified to the user and the "myData" page is shown with the available data \\ \hline
\end{tabular}
\end{table}

\begin{table}[]
\begin{tabular}{|l|p{12cm}|}
\hline
Name             & Create a Run \\ \hline
Actor            & User (Organizer) \\ \hline
Entry conditions & The organizer has previously installed the Track4Run application \\ \hline
Events flow      & \begin{enumerate}
\item The organizer clicks on the "+" icon on the bottom of the "myRuns" section
\item Enter information about the run
\item Select the run path from the map
\item Click on the "Create" button
\end{enumerate} \\ \hline
Exit conditions  & The organizer has successfully created a run \\ \hline
Exceptions       & \\ \hline
\end{tabular}
\end{table}

\begin{table}[]
\begin{tabular}{|l|p{12cm}|}
\hline
Name             & Run enroll \\ \hline
Actor            & User (Runner)\\ \hline
Entry conditions & The runner has previously installed the Track4Run application \\ \hline
Events flow      & \begin{enumerate}
\item The runner clicks on the "Search" tab 
\item Choose from the "Future" list of runs which one to enroll
\item Clicks on the "Enroll" button
\end{enumerate} \\ \hline
Exit conditions  & The user has successfully enrolled to the run \\ \hline
Exceptions       & \begin{enumerate}
\item The user doesn't possess a device able to Track the runner
\item The user is in a bad shape for a run, medical advice is suggested
\end{enumerate} This exceptions is notified to the client and the procedure goes on\\ \hline
\end{tabular}
\end{table}

\begin{table}[]
\begin{tabular}{|l|p{12cm}|}
\hline
Name             & Run "unsubscription" \\ \hline
Actor            & User (Runner) \\ \hline
Entry conditions & User already enrolled on a Run and he/she want to unsubscribe \\ \hline
Events flow      & \begin{enumerate}
\item Looks up under "My Runs" tab for the previously joined run
\item Clicks on the "Unsubscribe" button
\end{enumerate} \\ \hline
Exit conditions  & User has successfully unsubscribed from the run \\ \hline
Exceptions       & \\ \hline
\end{tabular}
\end{table}

\begin{table}[]
\begin{tabular}{|l|p{12cm}|}
\hline
Name             & Watch Run \\ \hline
Actor            & User (Spectator), Users (Runners) \\ \hline
Entry conditions & A spectator goes under the "Search" section of Track4Run  \\ \hline
Events flow      & \begin{enumerate}
\item Select a Run form the list of ongoing runs
\item Watch the map of the run with the real time GPS tracking of the runners
\end{enumerate} \\ \hline
Exit conditions  & Run ends or spectator exits from the application \\ \hline
Exceptions       & \begin{enumerate}
\item Runner has connection problems
\item Spectator has connection problems
\item Server has connection problems
\end{enumerate} All exception are notified to the spectator and the process goes on\\ \hline
\end{tabular}
\end{table}

\begin{table}[]
\begin{tabular}{|l|p{12cm}|}
\hline
Name             & Run modification \\ \hline
Actor            & User (Organizer) \\ \hline
Entry conditions & The organizer goes under the "myRuns" tab of the Track4Run application \\ \hline
Events flow      & \begin{enumerate}
\item Select the run you want to modify from the one presented in the timeline
\item Clicks on the "Edit" button
\item Modify the information of the run
\item Click on the "Confirm" button
\end{enumerate} \\ \hline
Exit conditions  & The organizer successfully modified the run and the participants are notified\\ \hline
Exceptions       &\\ \hline
\end{tabular}
\end{table}

\begin{table}[]
\begin{tabular}{|l|p{12cm}|}
\hline
Name             & Run deletion \\ \hline
Actor            & User (Organizer) \\ \hline
Entry conditions & The organizer goes under the "myRuns" tab of the Track4Run application \\ \hline
Events flow      & \begin{enumerate}
\item Select the run you want to modify from the one presented in the timeline
\item Clicks on the trash bin icon
\end{enumerate} \\ \hline
Exit conditions  & The organizer successfully deleted the run and the participants are notified\\ \hline
Exceptions       &\\ \hline
\end{tabular}
\end{table}

\subsection{Performance Requirements}
5 seconds react time
\subsection{Design Constraints}
\subsubsection{Standard compliance}
GDPR (general data protection regulation), term of service...
\subsubsection{Hardware limitations}

\subsubsection{Any other constraint}
\subsection{Software System Attributes}
\subsubsection{Reliability}
up 24/7
\subsubsection{Availability}
\subsubsection{Security}
\subsubsection{Maintainability}
\subsubsection{Portability}
