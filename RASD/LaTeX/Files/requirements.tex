Organize this section according to the rules defined in the project description. 
\subsection{External Interface Requirements}
\subsubsection{User Interface}
\subsubsection{Hardware Interfaces}
\subsubsection{Software Interfaces}
\subsubsection{Communication Interfaces}
\subsection{Functional Requirements}
\subsection{Scenarios}
\subsubsection{Scenario 1}
John tells Frank about a new service called Data4Help that helps you keeping track of your health status. Frank, interested in checking his heart beat rate, downloads the app on his smartphone. He fires up  Data4Help and he register to the service by filling a form with his personal information, including the fiscal code and a password of his choice that will be used as credentials fo the log-in. 
After filling the form he clicks on the final checkbox to accept the personal data treatment policy.
Right after clicking on the "Submit" button, Frank receives a confirm alert that he has successfully registered to Data4Help.
\subsubsection{Scenario 2}
Mary, a Data4Help customer, received for her birthday the new Eppol iClock. After the initial setting of the device, she decides to download the Data4Help app for her smartwatch. Once installed, she logs-in with her account and right after that, the "Device Management" view of the application appears, asking Mary to manage her smartphone and her smartwatch. Mary taps on the iClock icon and by selecting the "Heart Rate" from a dropdown she assigns the tracking of that paramater to her smartwatch. Finally, with the same procedure, she assigns to her smartphone the tracking of the "Position" parameter.
To confirm the settings, she click on the "Ok" button and the app saves the changes and returns to the home page.
\subsubsection{Scenario 3}
Dr. Lemmings, after a medical examination, suggests to Steve, one of his patients with diabete, to periodically check the glucose levels observed by his medical device connected with the Data4Help application. 
When he got home, Steve starts the Data4Help app from his smartphone and he clicks on the "myData" tab in the homepage. So a nice view appears, containing all the information about his monitored health parameters with a lot of colorful diagrams. Steve filters out the displayed data by selecting the "Glucose Level" radio button on the top of the page and he changes the information granularity using a dropdown from "Year" to  "Month". 
\subsubsection{Scenario 4}
Michael Garcia, Boyern CEO, a big pharmaceutical company, heard about Data4Help service and decided in agreement with the Administrative Board of the company to introduce the software service in the company. To do that he visits  www.trackme.com/data4help web page and goes in the B2B solutions to register his company as a registered Third Party of the service. When the page shows up he sees that he must fill in a form to complete the procedure. 
Michael fills the form with all the requested information about the company, including the VAT-number, an e-mail and a password. After the filling procedure he clicks on the "Submit" button on the bottom and after some seconds he receives a mail with a link to complete the registration procedure.
\subsubsection{Scenario 5}
Brian, patient of the Lenox Hill Private Medical Center, was released  yesterday.  The clinic, registered to Data4Help service, decides to monitor his health status to see if the treatment he was under reached the desired results.
The doctor that was in charge of Brian logs-in to the Data4Help page of his clinic. Once logged in, he inserts Brian's fiscal code and selects from a checkbox "Heart Beat" and "Temperature" as the data to monitor. Finally he chooses from the dropdown  "3 months" for the observation period.
As soon as the doctor clicks on the "Send request" button on the bottom of the page, Brian receives a notification on his smartphone. Brian opens the application, taps on the "Requests" tab in the home page, selects the mentioned request and accepts it. After that Lenox Hill clinic receives notification e-mail saying that the patient accepted the treatment of his personal data. As soon as the doctor sees that e-mail, he goes in the "Sent Requests" section of the Data4Help personal page and after clicking on the Brian aswer all the requested data are displayed on the screen.
\subsubsection{Scenario 6}
Pfuzer, a big pharmaceutical company registered to Data4Help service, needs to gather the heart rate data of all the italian young people under 30 years old for a market analysis aimed at evaluating the production of a new drug against heart disease. For this reason, the marketing manager of the company Todd Chavez goes to the Pfuzer personal Data4Help page and once logged  in he clicks on the "Aggregated requests" tab in the home page. Once in the page he types in the textbox "Italy" as location, and selects the age range "<= 30" using a slider. Then he clicks on "Send request" on the bottom of the page and an warning message is immediately displayed on screen saying that the request cannot be satisfied. Todd, decides to untighten the search parameters and after changing them the request is satisfied and the requested data is immediately displayed on screen. 
\subsection{Performance Requirements}
\subsection{Design Constraints}
\subsubsection{Standard compliance}
\subsubsection{Hardware limitations}
\subsubsection{Any other constraint}
\subsection{Software System Attributes}
\subsubsection{Reliability}
\subsubsection{Availability}
\subsubsection{Security}
\subsubsection{Maintainability}
\subsubsection{Portability}
